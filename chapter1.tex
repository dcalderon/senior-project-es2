\chapter{Introduction}
%\pagestyle{chapter}
\begin{shadequote}
The sure and definite determination (of species of bacteria) requires so much time, so much acumen of eye and judgement, so much of perseverance and patience that there is hardly anything else so \mbox{difficult}. \par\emph{Otto F. M\"uller}
\end{shadequote}


\section{Description of Problem}
The quote above, by no mistake, graced the cover of the International Journal of Systematic and Evolutionary Microbiology for decades.
Whereas plant and animal systematists are guided by a theory based approach to \index{demarcating} species, microbiologists have yet to agree on a set of ecological and evolutionary properties that could serve to identify bacterial species~\cite{cohan2007systematics}.
They are naturally handicapped by the paucity of morphological differences that could aide in differentiation of closely related bacterial species.
Another factor is that microbiologists cannot predict which traits will cause a speciation event since bacteria are capable of receiving genes from distant relatives through a process known as horizontal gene transfer (HGT)~\cite{cohan2007systematics}.
Thus, in order to effectively understand the microbiome we must strive towards developing a method for consistently demarcating groups, from bacterial diversity, that play distinct ecological roles~\cite{koeppel2008identifying}.


Initially, closely related bacterial species were identified based on metabolic phenotype.
Systematists now rely on molecular approaches that utilize the decreasing cost of DNA sequencing to compare genetic information.
A 70\% cutoff was established for whole genome hybridization studies (comparing loss and gain of large chunks of DNA), replaced by varying degrees of sequence identities in homologous genes~\cite{cohan2007systematics,carlo}.
From these technological breakthroughs scientists have taken great steps towards understanding bacteria speciation, at the same time they have brought into focus new difficulties.


\subsection{Diversity of Bacterial Species}
Estimates of eukaryotic diversity fall within the range of 10 to 50 million species. Even though we have only observed approximately 9000 prokaryotic species, indirect approaches that do not rely on cultivation hint at the existence of a billion or more prokaryotic species worldwide and 10 million within a given habitat~\cite{cohan2008origins}.
To observe biodiversity through molecular means, scientists should find organisms in highly distinct sequence clusters, since each cluster has had a long history of separate evolution they have likely evolved unique adaptations shared by the entire cluster~\cite{cohan2007systematics}.
The only rational approach to clustering such a large group of diverse organisms effectively is with a theory based molecular method.

Current protocols for deciding bacterial lineage are functionally incomplete. Recent ecological studies show that a named bacterial species is typically an assemblage of closely related but ecologically distinct populations~\cite{cohan2007systematics}.
Thus, within a named species established by antiquated methods, one may find distantly related organisms that do not naturally fit in with what we define to be a cohesive species cluster, which contradicts ideal approaches to biodiversity discovery.
Comprehensive study is impossible.

However, fortunately organisms from every known community appear to cluster into discrete ecologically interchangeable individuals~\cite{cohan2007systematics}.
Our envisioned demarcation algorithm would be capable of identifying putative clusters of ecologically distinct organisms within named bacterial clades.
While accuracy is of utmost importance, due to the large numbers of potential bacterial species we would appreciate an efficient demarcation algorithm.


\subsection{Differences of Bacterial Population Dynamics}
As briefly mentioned earlier there exist peculiarities of bacterial population dynamics that complicate demarcation.


\subsubsection*{Asexual reproduction}
Prokaryotes have the ability to reproduce clonally.
In fact, genetic exchange occurs typically through processes not tied to reproduction.
Thus, sexual isolation is not a pre-requisite for permanent divergence between distinct ecological bacterial populations ~\cite{cohan2007systematics}.
This means that sympatric speciation becomes a common occurrence.

\subsubsection*{Horizontal Gene Transfer (HGT)}

%\section{Benefits to Understanding Bacteria Speciation}
\section{Practical Applications}
We must first be able to identify the basic units operating within the system, in order to understand the microbiome.
Once we have established the atomic functional unit we can then start building collections of relationships between these units. 

\subsubsection*{Antibiotic resistance and other health benefits}
\subsubsection*{Simplify the burden of industrial testing of bacterial strains for their safety and efficacy in agricultural applications (from ``An Ecotype-based systematics'' there are some benefits discussion their)}
\subsubsection*{Bioremediation}
\subsubsection*{True quantification of ecological diversity within a community}


\section{Outline}% SHOULD I PUT THIS SECTION AT THE BEGINNING?

My aims for this project are several.
The final result should be a production ready version of the optimized Ecotypes Simulation (ES2) software.
However, first we plan on carrying out several tests to show ES2 is accurate, and efficient.
I will run ES2 through the same previous test parameters the Cohan lab used to test ES1 to compare all demarcation algorithms on small inputs.
This includes generated datasets and environment collected sequences.
It is important to demonstrate nearly identical accuracy scores of ES2 to ES1.
We hypothesized that the new, more efficient, demarcating algorithm within ES2 will not significantly affect the accuracy of our output.
Next, I want to identify the new reasonable upper limit for input size (i.e., number of sequences) for ES2.
Then I will run ES2 in conjunction with other available demarcation programs (ES1, BAPS, GMYC, AdaptML) on large generated datasets (up to the newly established reasonable limit) and evaluate demarcation algorithm accuracy and speed.

Chapter 1 will introduce the problem that ES tries to address, and the benefits that understanding bacterial speciation will achieve.
The following chapter will go over several current molecular models for bacteria speciation, the underlying algorithms that make them function, and describe the design of ES.
Next section discusses our approach to ES optimization, as well as other future enhancements planned.
In the fourth chapter, I will go over comparison results between various demarcation programs.
Finally, the conclusion will contain a discussion of what we set out to accomplish compared to what we achieved, shortcomings, and some final thoughts on future directions.

