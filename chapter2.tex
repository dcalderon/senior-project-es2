\chapter{Prior Work}
\section{Background and Theory}
\subsection{Ecotype Model of Bacterial Species}
An ecotype is defined as a bacterial cluster with individuals that are ecologically similar to one another, so much so that genetic diversity within the ecotype is limited by a cohesive force, either periodic selection or genetic drift, or both~\cite{cohan2007systematics}.
By linking species and ecological niche we take advantage of the natural clustering of organisms according to environmental resources.
Since these clusters are evolving together we do not need to utilize morphological differences and can instead use a molecular approach for demarcation.

Ecotypes have all the dynamic properties of a species, each is a cohesive group, different ecotypes are irreversibly separate because they are out of range of one another's cohesive force, and because recombination is too rare to prevent their adaptive divergence; they are by definition ecologically distinct, which allows them to coexist into the future~\cite{cohan2007systematics}.
Or, essentially the fundamental unit of microbial ecosystems.
Efforts to define prokaryotic species according to these properties have differed most profoundly in the forces of cohesion thought to be most important for prokaryotic species~\cite{cohan2008origins}.


\subsection{Stable Ecotype Model}
Outline the stable ecotype model and why it's relevant
\section{Algorithm}
Here I'll go over the process in non-system specific way. Make phylogeny based on distances. Then we want to find the parameters that describe the sequence identity graph. Then describe why these parameters are relevant and the power they give us. Allow us to pick at which point we pass diversity between different species groups and diversity within a species group.
\section{Ecotype Simulation}
Here we can show how we attempt to implement this algorithm. Use system diagrams going from binning to brute force to hill climbing to automatic demarcations. We can choose how to divide this up into sections later. One subsection per program?