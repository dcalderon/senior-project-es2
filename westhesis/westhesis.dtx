% \iffalse

%% westhesis:  Document class for Wesleyan University theses.
%% N. Danner (ndanner@wesleyan.edu)
%% Copyright 1999--2011

%% The following RCS info is from the iuthesis class upon which westhesis
%% is based.
%%
%% RCS Info:
%%      iuthesis.dtx,v 1.2.0.23 1999/06/04 14:30:00 ndanger Exp
%      iuthesis.dtx,v
%      Revision 1.2.0.23  1999/06/04 14:30:00  ndanger
%      Definition of \makededication when *abstract option given done with
%      \newcommand so ensure it swallows the optional argument (if present).
%
%      Revision 1.2.0.22  1999/06/04 14:22:35  ndanger
%      Third reader was missing period after Ph.D in ugs abstract.
%
%      Revision 1.2.0.21  1999/05/21 01:18:55  ndanger
%      New option 'single':  to single-space the thesis.
%
%      Revision 1.2.0.20  1999/05/21 01:13:28  ndanger
%      New option 'three', for when there are only three committee members.
%
%      Revision 1.2.0.19  1999/05/16 01:26:16  ndanger
%      Miscellaneous stuff to add support for Masters theses.
%
%      Revision 1.2.0.18  1999/05/16 01:07:42  ndanger
%      \baselinestretch set to \IUT@blstretch instead of 1.66; \IUT@blstretch
%      is set according to type size.
%
%      Revision 1.2.0.17  1999/05/14 04:03:50  ndanger
%      1.  By an overwhelming vote, the dedication is centered, near the
%          top of the page (exact distance from top specified by an
%          optional argument to \makededication).
%
%      2.  All references to 'sechead' option (supplanted by 'section'
%          pagestyle) removed.
%
%      Revision 1.2.0.16  1999/05/14 03:07:38  ndanger
%      Page styles:  Defined 'chapter', 'section', and 'chapsec' page
%      styles, all of which redefine \ps@headings, which is what is
%      ultimately used.
%
%      Revision 1.2.0.15  1999/05/12 16:07:18  ndanger
%      Abstract handling: 'abstractonly' option eliminated in favor of
%      'umiabstract' and 'ugsabstract'.  The former typesets the title page
%      and an abstract with the advisor's name after the abstract, the latter
%      just the abstract, possibly with signature lines.  This has
%      ramifications throughout the code.
%
%      Revision 1.2.0.14  1999/05/05 03:40:44  ndanger
%      Misc. documentation changes.
%
%      Revision 1.2.0.13  1999/05/05 02:58:43  ndanger
%      1.  New option 'sechead':  to use section name in running head.
%      2.  Miscellaneous documentation stuff.
%
%      Revision 1.2.0.12  1999/05/05 02:26:26  ndanger
%      Added documentation about Underfull \vboxes.
%
%      Revision 1.2.0.11  1999/05/05 02:13:38  ndanger
%      More abstract handling:
%        1.  \unvbox'ed the \abstractbox in case it needs a page break in the
%            middle.  Same for \dedicationbox and \acknowledgebox
%        2.  Typeset the abstract as a trivlist, just like the other
%            frontmatter pages.
%
%      Revision 1.2.0.10  1999/05/05 02:00:12  ndanger
%      Philosophy is spelled correctly on the title page (eesh).
%
%      Revision 1.2.0.9  1999/05/03 03:21:02  ndanger
%      Miscellaneous fixes:
%        1.  Checks to see that LaTeX is release 1996/12/01 or later (sometime
%            before then the implementation of \DeclareOption was changed) and
%            that AMSLaTeX is release 1995/01/27 (1.2, not 1.2b) or later
%            (handling of amstex/amsmath different).
%        2.  UMI now stands for University Microfilms International.
%        3.  Added a \fifthreader command, which can be optionally specified
%            by the user.
%        4.  Sprinkled %'s to get rid of some stray spaces.
%
%      Revision 1.2.0.8  1999/05/03 03:01:19  ndanger
%      Acknowledgements are double-spaced.
%
%      Revision 1.2.0.7  1999/05/03 02:59:35  ndanger
%      Abstract handling:
%        1.  If signature lines extend on to second page, that page is now
%            empty (put \newpage inside local group where \pagestyle command
%            given).
%        2.  \raggedbottom command given when 'abstractonly' specified;
%            otherwise, if signature lines extend to second page, the first
%            page is set flush bottom, which usually means lots of space
%            between the heading and abstract.
%
%      Revision 1.2.0.6  1999/05/03 01:42:49  ndanger
%      Signature boxes take an option argument to specify the width (default
%      3 inches), and no longer are put in a box the width of the page.
%
%      Signature lines on abstract arranged in two columns to save space.
%
%      Revision 1.2.0.5  1999/05/03 01:19:32  ndanger
%      Removed the leading 'Abstract' from abstract when 'abstractonly'
%      option given, so that line breaks are the same everywhere.
%
%      Revision 1.2.0.4  1999/04/25 04:17:44  ndanger
%      Added code to create iuthesis.tex, which can be used to generate
%      user documentation (i.e., without line-by-line code docs).
%
%      Revision 1.2.0.3  1999/04/25 04:01:47  ndanger
%      Added Dedication, Acknowledgements pages, fixed another bug in
%      Abstract handling.
%
%      Revision 1.2.0.2  1999/04/25 03:05:28  ndanger
%      Fixed up abstract handling:  sig lines only under 'abstractonly',
%      \makeabstract creates a head with \@makeschapterhead when being
%      typeset in the thesis.  Also added instructions for how to produce
%      the various kinds of abstracts RUGS wants.
%
%      Revision 1.2.0.1  1999/04/25 01:05:29  ndanger
%      Define \end@fterabstract to conditionally end document in \makeabstract
%      if 'abstractonly' option has been given.
%
%      Revision 1.2  1999/04/23 02:49:43  ndanger
%      Some more end-of-line comment chars in defnition of \check@val to
%      remove stray spaces.
%
%      Revision 1.1  1999/04/23 02:44:19  ndanger
%      Initial revision
%

% \fi

\def\fileversion{1.0}
\def\filedate{March 2011}

% \iffalse
%<*driver>
\documentclass{ltxdoc}
\usepackage{graphics}
\begin{document}
\DocInput{westhesis.dtx}
\end{document}
%</driver>
% \fi

% \iffalse
%<*docs>
%    \begin{macrocode}
\documentclass{ltxdoc}
\usepackage{graphics}
\OnlyDescription
\begin{document}
\DocInput{westhesis.dtx}
\end{document}
%    \end{macrocode}
%</docs>
% \fi

% \let\cls\textsf\let\pkg\textsf\let\option\textbf
% \title{The \cls{westhesis} Document Class}
% \author{N. Danner}
% \date{Version \fileversion, \filedate}
% \maketitle
%


%\section{Introduction}
%The \cls{westhesis} document class provides output for theses
%acceptable to Wesleyan University.  It currently only supports
%undergraduate Honors theses.
%It is based on the \cls{amsbook} document class
%and requires that the $\mathcal{AMS}$\LaTeX\ package version 1.2 or
%later be
%installed (version 1.2b will \emph{not} work).  
%One facet of this is that if you want to use
%\cls{westhesis} on different machines, you must ensure that you have
%the same version of \cls{amsbook} installed on both.  Otherwise,
%you may get different output, depending on the version installed.
%If you have a version of \LaTeX\ that predates the 1996/12/01 release,
%\cls{westhesis} will complain, but may still work.  If you have a version
%of $\mathcal{AMS}$\LaTeX\ that predates version 1.2, \cls{westhesis} will
%complain and crash on you.
%\cls{westhesis} provides many options, including
%printing signature lines on the abstract page,
%the ability to print just the abstract (without page numbers) for
%submission to University Microfilms International, and printing a
%message at the bottom of draft versions.

%\section{The Options}
%The following are the options that may be specified as optional
%arguments to the |\documentclass| line.
%\begin{description}

%\item[draft] This option causes a line a text to be printed at the
%bottom of each page; by default, the text is |DRAFT: <date>|, and it
%can be changed with the |\WTdraftmsg| command.  The \option{draft}
%option is also passed to \cls{amsbook}, which will cause it to mark
%overfull boxes.

%\item[final] This option supresses printing of the draft message and
%is also passed to \cls{amsbook} and all packages that you load.  Of
%course, it seems pointless since you might as well just not specify
%\option{draft} as an option, but it is here so that it can also be
%passed to packages that you may only want loaded during draft runs
%(e.g., if you use the \pkg{showkeys} package, it will only really
%be loaded if it is not passed the \option{final} option).

\iffalse
%\item[umiabstract] Print only the title page and abstract without page
%numbers for UMI (see Section~\ref{sec:misc}).  This option overrides
%the \option{noabstract} option.

%\item[ugsabstract] Print only the abstract without page numbers for
%the Graduate School (see Section~\ref{sec:misc}).  This option overrides
%the \option{noabstract} option.

%\item[abstractsigs] Print signature lines on the abstract page when 
%\option{ugsabstract} is specified.  This
%is not required by the Graduate School, but you may think it looks
%neater.

%\item[noabstractsigs] Do not print the signature lines on the abstract
%page.

%\item[noabstract] Do not print the abstract in the thesis.  Even if
%you don't want an abstract in your thesis, you still need to supply
%one to UMI; with this option, you can still put your abstract in your
%thesis file and typeset it using either the \option{umiabstract} 
%or \option{ugsabstract} options, but
%it will not appear in the thesis itself.

%%\item[ma] For a Master of Arts dissertation.

%%\item[ms] For a Master of Science dissertation.

%%\item[three] Use this if you only have three people on your committee
%%(including your advisor).
\fi

%\item[single] Single-space the thesis.  Of course, you cannot use this
%for the official copy, but it saves paper if you want to print out a
%copy for yourself.

%\end{description}

%All other options (e.g., \option{11pt}) are passed to \cls{amsbook}.
%The default option is \option{draft}.  The
%\option{oneside} option is automatically passed to \cls{amsbook}.

%\section{Using the \cls{westhesis} Document Class}

%Figure~\ref{fig:galley} shows a sample galley file using the
%\cls{westhesis} document class.  
%Here, we describe in order each of the commands and environments used.
%Of course, we start with a |\documentclass| command, which should need
%no additional explanation.  You should put your own |\usepackage| commands
%and local macro definitions between the |\documentclass| and |\title|
%commands.  Note that since \cls{amsbook} will be loaded automatically,
%and that will in turn load the \pkg{amsmath} package, you should not
%explicitly load the latter.


%\newbox\galleyleft\newbox\galleyright\newbox\galleybox
%\setbox\galleyleft\vbox{\hsize=3in%
%\begin{verbatim}
%\documentclass[11pt,draft]{westhesis}

%\title{A Proof that $\mathbf{P}\not=\mathbf{NP}$}
%\author{Turing A. Ward}

%\department{Mathematics and Computer Science}
%\submitdate{April 2011}
%copyrightyear{2011}

%\makeindex

%\begin{document}

%\begin{dedication}
%I dedicate this thesis to my future devotees.
%\end{dedication}

%\begin{acknowledgements}
%I acknowledge, therefore I am.
%\end{acknowledgements}
%\end{verbatim}
%}

%\setbox\galleyright\vbox{\hsize=3in%
%\begin{verbatim}
%\begin{abstract}
%In this thesis, we prove that $\math{P}\not=\mathbf{NP},
%except when $\mathbf P=0$ or $\mathbf N=1$.
%\end{abstract}

%\frontmatter
%\maketitle
%\makecopyright
%\makededication
%\makeack
%\makeabstract
%\tableofcontents

%\mainmatter

%\include{intro}
%\include{proof}

%\appendix
%\include{refutation}

%\bibliographystyle{abbrv}
%\bibliography{mybib}

%\printindex

%\backmatter
%\include{vita}

%\end{document}
%\end{verbatim}
%}

%\begin{figure}
%\rotatebox{90}{\box\galleyright}
%\vskip\baselineskip
%\hrule
%\vskip\baselineskip
%\rotatebox{90}{\box\galleyleft}
%\caption{A sample thesis galley file.}
%\label{fig:galley}
%\end{figure}

%\bgroup
%\parindent=0pt\parskip=\baselineskip

%\DescribeMacro{\title}\DescribeMacro{\author}
%The |\title| and |\author| command specify the title and
%author of your dissertation (presumably the latter is you).  These
%macros do \emph{not} take any optional arguments (for example, to
%specify short versions), unlike the corresponding commands in the
%\cls{amsbook} document class.

%\DescribeMacro{\advisor}\DescribeMacro{\secondreader}
%\DescribeMacro{\thirdreader}\DescribeMacro{\fourthreader}
%\DescribeMacro{\fifthreader}
%Use the |\advisor| and |reader| macros to specify your advisor and other 
%committee members' names.  In the signature lines, ``Ph.D.'' will be
%appended to their names.  If you print something with signature lines
%(like the signature page or an abstract with \option{abstractsigs} in
%effect) and you have not issued one of the commands other than
%|\fifthreader|, you will get complaints.  If you do not specify
%|\fifthreader|, then no signature line will be typeset for a fifth
%committee member.  These commands have no effect for Honors theses,
%but can be used as ``documentation.''

%\DescribeMacro{\department}\DescribeMacro{\submitdate}
%\DescribeMacro{\copyrightyear}\DescribeMacro{\departmentname}
%Use the |\department|, |\submitdate|, and |\copyrightyear| macros
%to specify the department in which you will receive your degree,
%when you actually hold the defense, and the year of the copyright
%(usually the year of graduation).  You do not need to specify the copyright
%year if you do not use |\copyrightpage| (copyright pages are not appropriate
%in an Honors thesis).  Use the |\departmentname| macro
%to specify what to call your ``department'' if something other than
% ``Department'' (e.g., if your thesis is in the School of Library and
% Information Sciences, you would have 
% |\department{Library and Information Sciences}| 
% and |\departmentname{School}| in your preamble so that the title page
% would say ``School of Library and Information Sciences'').

%\DescribeEnv{dedication}\DescribeEnv{acknowledgements}\DescribeEnv{abstract}
%Use these environments for the specified pieces of text.  The first two
%are optional.  
\iffalse
%While putting an abstract in your thesis is
%optional, you are required to submit two copies of the abstract to
%the Graduate School, so you will need to use the |abstract| environment.
\fi
%The Dedication is typeset on an empty page (except page number),
%centered, single-spaced, one inch from the top margin (but see
%|\makededication|, below).  The
%Acknowledgements and Abstract 
\iffalse
%(if you have not specified 
%\option{ugsabstract}, \option{umiabstract}, or \option{noabstract})
\fi
%are typeset double-spaced as unnumbered chapters.

%\DescribeMacro{\frontmatter}
%The |\frontmatter| command must come before any text is actually typeset
%(including the title page).  Primarily it sets up page numbering correctly.
%The title page will be unnumbered, and every other
%page that occurs between the
%|\frontmatter| and |\mainmatter| commands will be given page numbers in
%lower-case roman type, starting with page ii.  
\iffalse
%The order in which the next set of commands before
%the |\mainmatter| command are issued is specified by the Graduate School;
%they'll be unhappy if you do anything else (although currently the 
%document class will let you do whatever you want).
\fi

%\DescribeMacro{\maketitle}
%The |\maketitle| command prints the title page.  This page is formatted
%with your thesis title and name, and the miscellaneous wordiness at the
%bottom saying that you are submitting this in partial fulfillment of the
%requirements for your degree.

% %\DescribeMacro{\signaturepage}
%The |\signaturepage| command creates the page where your committee signs
%off on the thesis.  You should not have a signature page in an Honors
%thesis.

%\DescribeMacro{\copyrightpage}
%The |\copyrightpage| command creates the copyright page (go figure).
%You should not have a copyright page in an Honors thesis.

%\DescribeMacro{\makededication}
%\DescribeMacro{\makeack}
%The |\makededication| and |\makeack| macros typeset the Dedication and
%Acknowledgements page, provided you have specified the text with the
%|dedication| and |acknowledgements| environments previously.
%|\makededication| takes a single optional argument to specify the
%vertical distance from the top margin to typeset the dedication; default
%is one inch.

%\DescribeMacro{\makeabstract}
%Typeset the abstract.
\iffalse
%If you have not specified the \option{noabstract} option
%the |\makeabstract| command
%will typeset the abstract.  If you've specified either of
%the \option{ugsabstract} or \option{umiabstract}
%options, then you must still have the
%|\makeabstract| command to actually typeset the abstract.
\fi

%\DescribeMacro{\tableofcontents}
%The |\tableofcontents| command is the usual one.  It is here just to point
%out where it should appear (i.e., as front matter).  Likewise any other
%such lists, such as a table of figures, should also appear here.

%\DescribeMacro{\mainmatter}
%Put the |\mainmatter| command just before the actual text of your 
%dissertation.  Pages will be numbered in arabic starting over at page 1.
%Following the |\mainmatter| command will either be your
%thesis, or (more likely) a sequence of |\include| commands to include the
%files containing the various chapters of your thesis, as shown here.
%Also notice that appendices, bibliography, and the index (if you have any
%of them) are considered part of the main matter, and these are all created
%as usual.

\iffalse
%\DescribeMacro{\backmatter}
%The |\backmatter| command turns off page numbering.  The only material
%in your thesis that can be placed after the |\backmatter| command is
%your vita.
\fi

%\egroup

%\section{Miscellaneous Features and Notes}
%\label{sec:misc}

%\paragraph{Sectioning}
%Sectioning commands from |\part| downwards work as expected.  In general
%the header and footer are determined by the page style (see below), but
%the first page of a chapter always contains an empty header and the 
%page number centered in the footer.
%A bonus
%is that \cls{amsbook} (and therefore \cls{westhesis}) handles optional
%arguments to |\chapter| and |\section|
%commands intelligently:  the optional argument is
%used to specify the running head, in case your title is long; but
%the full title will be used in the table of contents.

% \paragraph{Page Styles}
% \cls{westhesis} provides three page styles that can be specified with
% the |\pagestyle| command:
% \begin{description}
% \item[|chapter|] The current chapter number and title will be centered in the
% header, the page number will be flush right in the header, and the
% footer will be empty.  This is the default, in case you do not have
% a |\pagestyle| command in your preamble.
% \item[|section|] The current section number and title will be centered
% in the header, the page number will be flush right in the header, and
% the footer will be empty.  If you have a |\chapter| with no |\section|
% command, then the chapter title will be centered in the header.
% \item[|chapsec|] ``|Chapter <chapter number>.<chapter title>|'' will 
% appear almost flush left and ``|<section number>.<section title>|'' will
% appear almost flush right in the header (the ``almost'' has to do with
% the way \cls{amsbook} handles this, which I haven't wanted to figure out),
% and the page number will be centered in the footer.  If you have a
% |\chapter| with no |\section| command, then by default the left- and
% right-hand sides of the header will specify the chapter title.  If you
% want an empty right-hand side, put |\sectionmark{}| immediately after
% your |\chapter| command.
% \end{description}

\iffalse
%\paragraph{Abstracts}
%Whether or not you want to put an abstract in your thesis, you will need
%to supply two abstracts to RUGS: one unsigned copy, which will go to
%UMI to be published
%in \emph{Dissertation Abstracts}, and one signed copy that will be kept on
%file by the Graduate School.  
%First, put the abstract in the
%|abstract| environment (even if you don't have a |\makeabstract| command).
%To make the unsigned copy that goes to UMI, specify the 
%\option{umiabstract} option to the |\documentclass| command; this will
%print your advisor's name after the abstract (bizarre, but UMI insists
%on it) and
%also produce a title page, which UMI also wants.  
%To make the signed copy that goes to the Graduate School, specify the
%\option{ugsabstract} option to the |\documentclass| command.
\fi

%\paragraph{Draft Message}
% \DescribeMacro{\draftmsg}\DescribeMacro{\WTdraftmsg}
% The message that appears in the footer can be defined using either the
% |\draftmsg| or |\WTdraftmsg| (the two are exactly the same)
% command, which uses its argument as the message.  For example,
% \begin{verbatim}
% \draftmsg{Draft Copy (Andrew Wiles):  \today}
% \end{verbatim}
%  will print
% \texttt{Draft Copy (Andrew Wiles): \today} in the footer, where
% \texttt{\today} will be replaced with the day that the file is processed.

%\paragraph{Underfull Pages}
%You are very likely to get a lot of |Underfull \vbox| messages when you
%typeset your thesis.  The \cls{amsbook} document class prevents a page
%from ending with the first line of a paragraph or beginning a page with
%the last line of a paragraph (whereas \LaTeX\ just discourages such
%behavior).  This means that lines have to be pushed around more often
%than usual, and on pages that have had lines moved off, the remaining
%lines need to be stretched some to fill out the space; thus the
%|Underfull \vbox| message.  Getting the vertical spacing right is further
%exacerbated by the fact that we must double-space the document.
%Alignment environments (like |eqnarray| and |align|) make things even
%worse, because they cannot be broken across page boundaries.  All these
%combine to make for a lot of pages that are underfull, and there seems
%to be no way around this.  If you really don't like how the lines are
%stretched out, you can specify |\raggedbottom| in the preamble of your
%document.  This will allow optimal spacing between lines, 
%but means that \LaTeX\ does not need to place the last line
%of each page at the same vertical position.

% \StopEventually{}

%\section{The Macros}

%\bgroup
%\parindent=0pt\parskip=.5\baselineskip

%    \begin{macrocode}
%<*class>
%    \end{macrocode}

% Specify what we need, what we provide, and identify ourselves to the log.
%    \begin{macrocode}
\NeedsTeXFormat{LaTeX2e}[1996/12/01] \ProvidesClass{westhesis}
\message{<< Class 'westhesis', v\fileversion, \filedate (N. Danner) >>}
%    \end{macrocode}

%\begin{macro}{\ifWT@debugging}
%\begin{macro}{\WT@debuggingtrue}
%\begin{macro}{\WT@debuggingfalse}
%Some miscellaneous macros for debugging and displaying what options 
%have been chosen in the |\documentclass| command.
%    \begin{macrocode}
\newif\ifWT@debugging \WT@debuggingtrue \WT@debuggingfalse
\def\WT@dbgmsg#1{\ifWT@debugging\message{#1}\fi}
\def\WT@optionmsg#1{\message{  westhesis option: #1}}
%    \end{macrocode}
%\end{macro}\end{macro}\end{macro}

%Now start handling the options, along with any additional macros,
%conditionals, etc. they may rely upon.  
%\begin{macro}{\draftmsg}
%\begin{macro}{\WTdraftmsg}
%\begin{macro}{\WT@showdraftfoot}
%|\draftmsg| and |\WTdraftmsg| (they are the same--the latter is for
%compatibility) are user commands 
%for specifying what should be printed
%in the footer when the \option{draft} option is given, and
%|\WT@showdraftfoot| is what is actually printed.  Initially, they
%are no-op and empty, so that unless the \option{draft} option is given,
%no such line will displayed, even if the |\WTdraftmsg| command is given
%in the preamble.  We also pass the \option{draft} option to
%\cls{amsbook}.
%    \begin{macrocode}
\def\WTdraftmsg#1{}
\let\draftmsg\WTdraftmsg
\def\WT@showdraftfoot{}
\DeclareOption{draft}{
  \WT@optionmsg{draft}
  \def\WTdraftmsg#1{\gdef\WT@showdraftfoot{#1}}
  \let\draftmsg\WTdraftmsg
  \def\WT@showdraftfoot{DRAFT: \today}
  \PassOptionsToClass{draft}{amsbook}
}
%    \end{macrocode}
%\end{macro}
%\end{macro}
%\end{macro}

%The \option{final} option causes the previous two macros to be a no-op
%and emtpy, respectively, and is also passed to \cls{amsbook}.
%    \begin{macrocode}
\DeclareOption{final}{
  \WT@optionmsg{final}
  \def\WTdraftmsg#1{}
  \let\draftmsg\WTdraftmsg
  \def\WT@showdraftfoot{}
  \PassOptionsToClass{final}{amsbook}
}
%    \end{macrocode}

%\begin{macro}{\WT@abstype}
%\begin{macro}{\WT@nmlabstract}
%\begin{macro}{\WT@umiabstract}
%\begin{macro}{\WT@ugsabstract}
%Although not currently enabled in \pkg{westhesis}, we will eventually
%provide support for Ph.D. theses.  For that, we may have different
%abstract types.  For example, the package on which this one is based had two
%other abstract types (one for UMI, one for the Indiana University
%Graduate School), each of which had its own format and also had to be
%available as a separate document.  When the appropriate type is enabled,
%the abstract is formatted appropriately, and if it has to be its own
%document, processing is stopped after typesetting the abstract.
% %The \option{umiabstract} and \option{ugsabstract}
% %options specify to typeset only the abstract
% %with no page numbers.  This will be implemented by
% %condtionally defining every |\frontmatter| macro as a no-op and 
% %ending the document at the end of the abstract
% %when this option is given, and otherwise
% %defining those macros reasonably.  If we are doing an abstract only,
% %then we need the |\raggedbottom| declaration to ensure that if
% %signature lines extend onto a second page, the first page is not
% %set flushbottom, which would typically leave a lot of space between
% %the heading and the abstract itself.
%    \begin{macrocode}
\newcount\WT@abstype
\newcount\WT@nmlabstract\WT@nmlabstract=0
\newcount\WT@umiabstract\WT@umiabstract=1
\newcount\WT@ugsabstract\WT@ugsabstract=2
\WT@abstype=\WT@nmlabstract
\DeclareOption{umiabstract}{
  \WT@optionmsg{umiabstract}
  \WT@abstype=\WT@umiabstract
  \AtEndOfClass{\raggedbottom}
}
\newif\ifWT@ugsabs
\WT@ugsabsfalse
\DeclareOption{ugsabstract}{
  \WT@optionmsg{ugsabstract}
  \WT@abstype=\WT@ugsabstract
  \AtEndOfClass{\raggedbottom}
}
%    \end{macrocode}
%\end{macro}
%\end{macro}

%\begin{macro}{\ifWT@abstractsigs}
%Do we have signature lines on the abstract page?  The default
%is \option{noabstractsigs},
%since it is not expected for Honors theses.
%    \begin{macrocode}
\newif\ifWT@abstractsigs
\DeclareOption{abstractsigs}{
  \WT@optionmsg{abstractsigs}
  \WT@abstractsigstrue
}
\DeclareOption{noabstractsigs}{
  \WT@optionmsg{noabstractsigs}
  \WT@abstractsigsfalse
}
%    \end{macrocode}
%\end{macro}

%\begin{macro}{\ifWT@noabstract}
%Should we typeset the abstract?  This is overridden if the
%one of the abstract-only options is given.
%    \begin{macrocode}
\newif\ifWT@noabstract
\WT@noabstractfalse
\DeclareOption{noabstract}{
  \WT@optionmsg{noabstract}
  \WT@noabstracttrue
}
%    \end{macrocode}
%\end{macro}

% % \option{ms} for Master of Science, \option{ma} for Master of Arts.
% % |\degree| is called when |\@degree| is initially defined, so we just
% % override it at the end of class loading.
% %    \begin{macrocode}
% \DeclareOption{ms}{
%   \AtEndOfClass{\degree{Master of Science}}
% }
% \DeclareOption{ma}{
%   \AtEndOfClass{\degree{Master of Arts}}
% }
% %    \end{macrocode}

% \begin{macro}{\ifWT@three}
% This option must be used if there are only three readers on the thesis.
% This is irrelevant for Honors theses, because the readers are not
% used in typesetting the thesis.
%    \begin{macrocode}
\newif\ifWT@three
\DeclareOption{three}{\WT@threetrue}
%    \end{macrocode}
% \end{macro}

% Make the document single-spaced.
%    \begin{macrocode}
\DeclareOption{single}{
  \AtEndOfClass{\def\WT@blstretch{1}}
}
%    \end{macrocode}

%Turn on debugging output.
%    \begin{macrocode}
\DeclareOption{debug}{
  \WT@debuggingtrue
}
%    \end{macrocode}

%Set everything up.
%    \begin{macrocode}
\DeclareOption*{\PassOptionsToClass{\CurrentOption}{amsbook}}
\ExecuteOptions{noabstractsigs,draft}
\ProcessOptions

\PassOptionsToClass{oneside}{amsbook}
\LoadClass{amsbook}[1995/01/27]
%    \end{macrocode}

%\paragraph{Page layout}  
%We need 1.5 inch margins on the left, 1.25 inch on the right,
%1 inch top and bottom.
%That means 1 inch \emph{above} the top of the header and
%\emph{below} the bottom of the footer (which will have a page number
%on first pages of chapters).  We start with a text height of 8.9 inches
%to get a smidgen of extra space at the bottom so that it doesn't look
%like the text is falling off of the page.  |\headheight| is increased
%by 2 points because oddly enough \cls{amsbook} v1.2 makes it a little too
%small for the default type size for the headers.
%    \begin{macrocode}
\topmargin 0pt
\oddsidemargin=.5in
\evensidemargin=\oddsidemargin
\advance\headheight 2pt
\textwidth 5.75in
\textheight 8.9in
\advance\textheight by -\headheight
\advance\textheight by -\headsep
\advance\textheight by -\footskip
\marginparwidth 0.5in
%    \end{macrocode}

%\paragraph{Page styles}
% Each |\ps@xxx| command just redefines |\ps@headings|, which is what will
% ultimately be called by |\mainmatter|, as well as setting up the marking
% commands.

% |\ps@chapter| has the chapter title centered, page number flush right
% in the header and empty footer; |\partmark| and |\sectionmark| are no-ops.
%    \begin{macrocode}
\def\ps@chapter{
  \gdef\ps@headings{
    \def\@oddhead{
      \normalfont\scriptsize\hfil\rightmark{}{}\hfil \llap{\thepage}
    }
    \def\@oddfoot{
      \normalfont\ttfamily\scriptsize\rlap{\WT@showdraftfoot\hfill}\hfill
    }
    \global\let\@evenfoot\@oddfoot
    \global\let\@mkboth\markboth
    \global\let\partmark\@gobble
    \gdef\chaptermark{%
      \@secmark\markboth\chapterrunhead{}}%
    \global\let\sectionmark\@gobble
  }
}
%    \end{macrocode}

% |\ps@section| has the section title centered, page number flush right
% in the header and empty footer;  |\partmark| is a no-op, but |\chaptermark|
% still works in case we have a chapter with no sections (in which case the
% chapter title goes in the header).
%    \begin{macrocode}
\def\ps@section{
  \gdef\ps@headings{
    \def\@oddhead{
      \normalfont\scriptsize\hfil\rightmark{}{}\hfil \llap{\thepage}
    }
    \def\@oddfoot{
      \normalfont\ttfamily\scriptsize\rlap{\WT@showdraftfoot\hfill}\hfill
    }
    \global\let\@evenfoot\@oddfoot
    \global\let\@mkboth\markboth
    \global\let\partmark\@gobble
    \gdef\chaptermark{%
      \@secmark\markboth\chapterrunhead{}}%
    \gdef\sectionmark{%
      \@secmark\markboth\sectionrunhead{}}%
  }
}
%    \end{macrocode}

% |\ps@chapsec| puts ``Chapter number.title'' almost flush left and
% the section number and title almost flush right (the ``almost'' is an
% \cls{amsbook} thing which I haven't wanted to figure out) in the header
% and the page number centered in the footer.  |\partmark| is a no-op,
% and |\sectionmark| uses our own |\WT@sectionrunhead| in case we want
% an empty section title in the header (see below).  Notice we use
% |\chaptername| as the first argument to |\chapterrunhead|; otherwise
% the header would not tell the reader which is the chapter and which
% is the section.
%    \begin{macrocode}
\def\ps@chapsec{
  \gdef\ps@headings{
    \def\@oddhead{
      \normalfont\scriptsize\rlap{\leftmark{}{}}\hfill\llap{\rightmark{}{}}
    }
    \def\@oddfoot{
      \normalfont\scriptsize\rlap{\ttfamily\WT@showdraftfoot\hfill}%
      \hfil\thepage\hfil
    }
    \global\let\@evenfoot\@oddfoot
    \global\let\@mkboth\markboth
    \global\let\partmark\@gobble
    \gdef\chaptermark{%
      \@secmark\markboth\chapterrunhead\chaptername}%
    \gdef\sectionmark{
      \@secmark\markright\WT@sectionrunhead{}}
  }
}
%    \end{macrocode}

% Either a bug or a feature with |\ps@chapsec|: a chapter with no
% sections will say Chapter number.title on both the left and right of
% the header.  |\sectionrunhead| will always print the chapter number,
% even if the section title is empty.  |\WT@sectionrunhead| tests its
% third argument (the section title), and only typesets something if it
% is non-empty.  If the user wants an empty right-hand side with
% |\ps@chapsec|, s/he just needs to say |\sectionmark{}| immediately
% after the |\chapter| command.
%    \begin{macrocode}
\def\WT@sectionrunhead#1#2#3{%
  \def\@tempa{#3}%
    \ifx\@empty\@tempa\else%
    \@ifnotempty{#2}{\uppercase{#1 #2}\@ifnotempty{#3}{. }}%
    \ifx\@empty\@tempa\else\uppercasenonmath\@tempa\@tempa\fi%
  \fi%
}
%    \end{macrocode}

% For main matter pages that are the first page of a chapter and front matter
% pages we put nothing in the header and center the page number in the
% footer.  The draft message also goes in the footer (flush left), but isn't
% allowed to take up any horizontal space.
%    \begin{macrocode}
\def\ps@plain{\ps@empty
  \def\@oddfoot{
    \normalfont\scriptsize
    \rlap{\ttfamily\WT@showdraftfoot}\hfil\thepage\hfil
  }
  \let\@evenfoot\@oddfoot
}
%    \end{macrocode}

%\paragraph{Preamble Commands}
%There will only be one author, and his/her name will only appear 
%on the title page and only in full form, so there is no need for the
%short form or andifying that \cls{amsbook} does.  Nothing interesting
%with |\title| is currently done by \cls{amsbook}, but we make sure
%that nothing interesting ever happens, except that we immediately
%convert it into uppercase.  The reason is that we may need an uppercase
%version twice (for \option{umiabstract}), and it seems that
%|\uppercasenonmath| hangs if you use it a second time on the same
%argument.  The |\advisor| and |\*reader| commands may be
%used in an Honors thesis, but they are not typeset anywhere.
%    \begin{macrocode}
\renewcommand{\author}[1]{\gdef\@author{#1}}\let\@author\relax
\renewcommand{\title}[1]{\gdef\@title{#1}}
\let\@title\relax

\def\submitdate#1{\gdef\@submitdate{#1}}\let\@submitdate\relax
\def\department#1{\gdef\@department{#1}}\let\@department\relax
\def\departmentname#1{\gdef\@departmentname{#1}}\departmentname{Department}
\def\degree#1{\gdef\@degree{#1}}\degree{Doctor of Philosophy}

\def\advisor#1{\gdef\@principaladvisor{#1}}\let\@principaladvisor\relax
\def\secondreader#1{\gdef\@secondreader{#1}}\let\@secondreader\relax
\def\thirdreader#1{\gdef\@thirdreader{#1}}\let\@thirdreader\relax
\def\fourthreader#1{\gdef\@fourthreader{#1}}\let\@fourthreader\relax
\def\fifthreader#1{\gdef\@fifthreader{#1}}\let\@fifthreader\relax

\def\copyrightyear#1{\gdef\@copyrightyear{#1}}
\let\@copyrightyear\relax
%    \end{macrocode}

%\paragraph{Front Matter Commands}~\\

% The dedication is to be centered; vertical positioning is handled by
% |\makededication|.

%    \begin{macrocode}
\newbox\dedicationbox
\newenvironment{dedication}{%
  \global\let\dedication\relax%
  \bgroup
  \renewcommand{\baselinestretch}{1}
  \normalsize
  \global\setbox\dedicationbox\vbox\bgroup%
  \begin{center}
}{
  \end{center}
  \egroup\egroup
}
%    \end{macrocode}

% The acknowledgements is typeset double-spaced as a trivlist.

%    \begin{macrocode}
\newbox\acknowledgebox
\newenvironment{acknowledgements}{%
  \global\let\acknowledgements\relax%
  \bgroup
  \renewcommand{\baselinestretch}{\WT@blstretch}
  \normalsize
  \global\setbox\acknowledgebox\vbox\bgroup
  \trivlist%
    \item[]\ignorespaces
}{
  \endtrivlist
  \egroup\egroup
}
%    \end{macrocode}

% The |abstract| environment typesets the abstract double-spaced in a box.
%    \begin{macrocode}
\renewenvironment{abstract}{%
  \global\let\abstract\relax%
  \bgroup%
  \renewcommand{\baselinestretch}{\WT@blstretch}%
  \normalsize%
  \global\setbox\abstractbox\vbox\bgroup
  \trivlist\item[]\ignorespaces
}%
{
  \endtrivlist
  \egroup\egroup
  \global\let\endabstract\relax
}
%    \end{macrocode}

% We do not define |\makeabstract| directly, because of the possible
% interaction between the abstract-only and \option{noabstract}
% options (which are not currently available); 
% the former must override the latter.  
%% If \option{ugsabstract} or \option{umiabstract}
%% has been specified, then the abstract should be typeset with no page numbers.
% The abstract is numbered as front matter with the author and title at the
% top.  If we are being typeset in the thesis, create the header with
% |\@makeschapterhead|, because |\chapter*| puts in a table-of-contents
% entry.  |\unvbox| the |\abstractbox| because otherwise we have problems
% with abstracts that would need a page break (this can happen with a
% 350 word abstract).  
%% If \option{umiabstract} or \option{ugsabstract}
%% has been specified, print the advisor's name or the signature lines,
%% respectively.  This could be done with an |\ifcase|, but seems less
%% transparent when re-reading the code.
% Conditionally define |\end@fterabstract| to end the document if
% abstract-only option has been specified, no-op otherwise, and then
% immediately execute |\end@fterabstract| to end the document in the former
% case.  If we just conditionally do an
% |\end{document}|, then we end the document before terminating the
% conditional.  We need to do a |\newpage| before ending the local group
% so that the |\pagestyle| declaration takes effect when typesetting an
% abstract only.  We also typeset the signature lines (if present)
% in two columns to save space.
%    \begin{macrocode}
\def\WT@defineabstract{
  \gdef\makeabstract{
    \typeout{Abstract}
    \bgroup
    \normalfont
    \ifnum\WT@abstype>0%
      \WT@dbgmsg{makeabstract: Setting abstract pagestyle empty}%
      \pagestyle{empty}\thispagestyle{empty}%
    \else%
      \WT@dbgmsg{makeabstract: Setting abstract pagestlye plain}%
    \fi%
    \ifnum\WT@abstype>0%
      \WT@dbgmsg{makeabstract: Setting title}%
      \begin{center}%
      \check@val\@author \\[.5\baselineskip]
%      \uppercasenonmath\@title\@title
      \check@val\@title
      \end{center}%
      \vskip\baselineskip
      \WT@dbgmsg{makeabstract: Done}%
    \else%
      \WT@dbgmsg{makeabstract: Setting title}%
      \@makeschapterhead{Abstract}
    \fi%
    \WT@dbgmsg{makeabstract: Setting abstract}%
    \unvbox\abstractbox
    \ifnum\WT@abstype=\WT@umiabstract%
      \WT@dbgmsg{makeabstract (umi): advisor's name}%
      \vskip\baselineskip%
      \hbox to\hsize{\hfill\check@val\@principaladvisor, Ph.D.}%
    \fi
    \ifnum\WT@abstype=\WT@ugsabstract%
      \WT@dbgmsg{makeabstract (ugs): signature lines}%
      \ifWT@abstractsigs%
      \vbox{
	\hbox to\textwidth{%
	  \WT@sig[2.5in]{\check@val\@principaladvisor, Ph.D.}\hfill%
	  \WT@sig[2.5in]{\check@val\@thirdreader, Ph.D.}%
	}%
	\hbox to\textwidth{%
	  \WT@sig[2.5in]{\check@val\@secondreader, Ph.D.}\hfill%
	  \ifWT@three\else\WT@sig[2.5in]{\check@val\@fourthreader, Ph.D.}\fi%
	}%
	\ifx\@fifthreader\relax\else%
	  \hbox to\textwidth{%
	    \hfill\WT@sig[2.5in]{\check@val\@fifthreader, Ph.D.}%
	  }%
	\fi%
      }
      \fi
    \fi
    \newpage
    \egroup
    \ifnum\WT@abstype>0
      \def\end@fterabstract{\end{document}}
    \else
      \def\end@fterabstract{}
    \fi
    \end@fterabstract
  }
}
%    \end{macrocode}

% Now we define |\makeabstract| provided that we have not specified
% \option{noabstract} and that option has not been overridden by
% \option{umiabstract} or \option{ugsabstract}.
%    \begin{macrocode}
\ifWT@noabstract
  \ifnum\WT@abstype>0\WT@defineabstract\else\def\makeabstract{}\fi
\else
  \WT@defineabstract
\fi
%    \end{macrocode}

% For \option{umiabstract}, legal front matter is the titlepage and
% abstract.  For \option{ugsabstract}, legal front matter is just
% the abstract.  For the thesis, all front matter is legal.
% ``Illegal'' here means we set the command to be a no-op, so that
% the user doesn't have to change the preamble ever, just the
% |\documentclass| options.  First the title page.
%    \begin{macrocode}
\ifnum\WT@abstype>1
  \WT@dbgmsg{Setting maketitle noop}
  \def\maketitle{}
\else
%    \end{macrocode}
%The title is is formatted according to the Honors Program specifications.
%    \begin{macrocode}
  \def\maketitle{
    \bgroup
	\thispagestyle{empty}
	\hrule\vskip.5ex
	\noindent Wesleyan University\hfill The Honors College
	\vskip.5ex\hrule
	\vskip3in
	\bgroup
		\Large
		\begin{center}\check@val\@title\end{center}
	\egroup
	\vskip.25in
	\bgroup
		\large
		\begin{center}
			by \\[\baselineskip]
			\check@val\@author
		\end{center}
	\egroup
	\vfill
	\bgroup
		\large
		\begin{center}
		A thesis submitted to the\\
		faculty of Wesleyan University\\
		in partial fulfillment of the requirements for the\\
		Degree of Bachelor of Arts\\
		with Departmental Honors in \check@val\@department
		\end{center}
		\vskip\baselineskip
		\hrule
		\vskip.5ex
		\noindent Middletown, Connecticut\hfill\check@val\@submitdate
		\vskip.5ex
		\hrule
	\egroup
	%\vskip.5in
	\egroup
	\newpage
  }
\fi
%    \end{macrocode}

% Now the front matter.  If we are just typesetting the abstract, then
% we make the front matter commands that come before |\makeabstract|
% no-ops; |\makededication| is done specially because it takes an
% optional argument which we must ensure is swallowed (otherwise it
% will appear in the beginning of the text of the abstract).
%    \begin{macrocode}
\ifnum\WT@abstype>0
  \WT@dbgmsg{Setting frontmatter commands noops}
  \def\signaturepage{}
  \def\copyrightpage{}
  \newcommand{\makededication}[1][]{}
  \def\makeack{}
\else
%    \end{macrocode}
% The
% signature page has the usual text, plus four signature lines stacked
% in a vertical list, followed by the submission date.
%    \begin{macrocode}
  \def\signaturepage{
    \typeout{Signature Page}
    \bgroup
    \noindent
    Accepted by the Graduate Faculty, Indiana University, in partial
    fulfillment of the requirements for the degree of \check@val\@degree.
    \egroup
    \vskip.5in
    \hbox to\textwidth{\hfill\WT@sig{\check@val\@principaladvisor, Ph.D.}}%
    \hbox to\textwidth{\hfill\WT@sig{\check@val\@secondreader, Ph.D.}}%
    \hbox to\textwidth{\hfill\WT@sig{\check@val\@thirdreader, Ph.D.}}%
    \ifWT@three\else%
      \hbox to\textwidth{\hfill\WT@sig{\check@val\@fourthreader, Ph.D.}}%
    \fi%
    \ifx\@fifthreader\relax\else%
      \hbox to\textwidth{\hfill\WT@sig{\check@val\@fifthreader, Ph.D.}}%
    \fi
    \vfill
    \noindent\check@val\@submitdate
    \vfill
    \newpage
  }
%    \end{macrocode}

% The copyright page does not seem to require much comment.
%    \begin{macrocode}
  \def\copyrightpage{
    \typeout{Copyright Page}
    \hbox{}\vfill
    \begin{center}
    Copyright \check@val\@copyrightyear \\
    \check@val\@author \\
    ALL RIGHTS RESERVED
    \end{center}
    \vfill
    \newpage
  }
%    \end{macrocode}

% The dedication is untitled and set one inch from the top margin, unless
% a different vertical skip is specified by the optional argument to
% |\makededication|.
%    \begin{macrocode}
  \newcommand{\makededication}[1][1in]{
    \ifvoid\dedicationbox\else
    \typeout{Dedication}
    \hbox{}\vskip#1\unvbox\dedicationbox\vfill%
    \newpage
    \fi
  }
%    \end{macrocode}

% The acknowledgements are typeset as an unnumbered chapter.  But
% |\chapter*| puts an entry in the table of contents, so we use
% |\@makeschapterhead|.

%    \begin{macrocode}
  \def\makeack{
    \ifvoid\acknowledgebox\else
    \typeout{Acknowledgements}
    \@makeschapterhead{Acknowledgements}
    \unvbox\acknowledgebox
    \newpage
    \fi
  }
%    \end{macrocode}

% End the conditional definition of the pages that might come before
% the abstract.
%    \begin{macrocode}
\fi
%    \end{macrocode}

% Each signature line is its own box, one inch heigh with width specified
% by the option argument to |\WT@sig| (default |3in|) (no
% depth).
%    \begin{macrocode}
\newcommand{\WT@sig}[2][3in]{
   \vbox{%
    \hrule width 0pt height 1in depth 0pt%
    \hrule width #1 height .4pt depth 0pt%
    \vskip2mm%
    \hbox to #1{\hfill #2}%
   }
}
%    \end{macrocode}

% Keep the table of contents simple, and make sure it is double spaced.
% Similar for list of tables and figures.
%    \begin{macrocode}
\def\tableofcontents{
  \bgroup
  \renewcommand{\baselinestretch}{\WT@blstretch}
  \normalfont
  \@starttoc{toc}\contentsname
  \egroup
}
\def\listoffigures{
  \bgroup
  \renewcommand{\baselinestretch}{\WT@blstretch}
  \normalfont
  \@starttoc{lof}\listfigurename
  \egroup
}
\def\listoftables{
  \bgroup
  \renewcommand{\baselinestretch}{\WT@blstretch}
  \normalfont
  \@starttoc{lot}\listtablename
  \egroup
}
%    \end{macrocode}

% Front matter has page numbers at the bottom in lower roman type.
%    \begin{macrocode}
\def\frontmatter{\cleardoublepage\pagenumbering{roman}\pagestyle{plain}}
%    \end{macrocode}

% Main matter uses |\headings| page style, arabic numbering, and double
% spacing.
%    \begin{macrocode}
\def\mainmatter{
  \cleardoublepage
  \pagenumbering{arabic}
  \pagestyle{headings}
  \renewcommand{\baselinestretch}{\WT@blstretch}
  \normalfont
}
%    \end{macrocode}

% Back matter (for vita only) gets empty page style (no numbering) and
% single spacing.
%    \begin{macrocode}
\def\backmatter{
  \newpage
  \pagestyle{empty}
  \renewcommand{\baselinestretch}{1}
  \normalfont
}
%    \end{macrocode}

% \paragraph{Utility Macros}
% |\check@val| checks to see that its argument has been defined; if so,
% it is used, otherwise a warning is issued and ??? printed.  Those
% comment markers at the end of lines are necessary--otherwise we get
% stray spaces.
%    \begin{macrocode}
\def\check@val#1{%
  \ifx#1\relax%
    \typeout{}%
    \typeout{!!!!!!!!}%
    \typeout{Warning: #1 not set!}%
    \typeout{!!!!!!!!}%
    \hbox{???}%
  \else%
    #1%
  \fi%
}
%    \end{macrocode}

% The 'see' command for the index should say ``see'', not ``see also''.
%    \begin{macrocode}
\renewcommand{\seename}{see}
%    \end{macrocode}

% The |\baselinestretch| is set according to typeset (LC, page 53).
% |\@mainsize| is an AMS macro that gives the (multiple digit) main font size.
% This may be overriden by the \option{single} option.
%    \begin{macrocode}
\def\WT@blstretch{1.67}
\ifnum\@mainsize=10\def\WT@blstretch{1.67}\fi
\ifnum\@mainsize=11\def\WT@blstretch{1.62}\fi
\ifnum\@mainsize=12\def\WT@blstretch{1.66}\fi
\WT@dbgmsg{Baseline stretch: \WT@blstretch}
%    \end{macrocode}

% Initialize things.
%    \begin{macrocode}
\pagestyle{chapter}
\pagenumbering{arabic}
\normalsize
%    \end{macrocode}

%    \begin{macrocode}
%</class>
%    \end{macrocode}

%\egroup

% \Finale
