\gobbletocpage
\chapter{Conclusions and Future work}
\restoretocpage

%\pagestyle{chapter}
\begin{shadequote}
The last ever dolphin message was misinterpreted as a surprisingly sophisticated attempt to do a double-backwards-somersault through a hoop whilst whistling the `Star Spangled Banner', but in fact the message was this: \mbox{\emph{So long and thanks for all the fish}}. \par--\emph{Douglas Adams}
\end{shadequote}

%or
%Flying is learning how to throw yourself at the ground and miss.
%or
%I love deadlines. I like the whooshing sound they make as they fly by.
%or
%There is a theory which states that if ever anyone discovers exactly what the Universe is for and why it is here, it will instantly disappear and be replaced by something even more bizarre and inexplicable.There is another theory which states that this has already happened.

%\section{Final remarks on ES}
\section{Remarks on ES}
%Re-iterate motivation and purpose of bacterial demarcations
Demarcating bacterial species is difficult.
However the path to understanding microbial speciation has many questions and benefits that attract the cleverest minds.
Do we realize how much microscopic diversity there is out there?
How can we use knowledge of microbial ecosystems to our benefit?
So far people have found applications in a variety of fields: epidemiology, biotechnology, bioremediation, and much more.
Ecotypes Simulation is a a step in the tradition of attempting to understand microbial ecosystem dynamics.

%Talk a little bit about the background and ES
It's based on ecotypes defined as a phylogenetic group of close relatives that are ecologically interchangeable, in that the members of an ecotype share genetic adaptations to a particular set of habitats, resources, and conditions; also, ecotypes are ecologically distinct from one another.
They are the fundamental unit of micro environments that we want to identify so we can understand relationships between groups of atomic units.

Ecotype Simulation uses the Stable Ecotype model that hypothesizes various ecotype dynamics such as periodic selection (diversity quashing events), ecotype formation (speciation events), and drift.
This is translated directly into the ES algorithm with its use of $npop$ (number of ecotypes), $\Omega$ (rate of periodic selection), and $\sigma$ (rate of ecotype formation).
ES1 first characterizes the community's observed evolutionary history, the through simulations estimates parameters, resulting in a maximum likelihood parametric characterization of the observed history.
Finally, we demarcate the individual strains into ecotypes.

%Talk a little bit about the improvements, resulting in ES2
ES2 improves on ES1's core algorithm by using an ultra metric tree when comparing simulated evolutionary histories to the observed history removing a $O(n^3)$ binning step.
During demarcation we also conduct Hillclimbing at each subtree; maintaining high quality parameter estimations as we recursively descend the phylogeny.
We are exploring including OpenMP parallelization; leading to a 50\% reduction in run time, when utilizing 8 threads.

%Summarize ES2 vs ES and ES2 vs all results
First I showed how close ES1 to ES2 accuracy (VI) scores were on a data set of 50 individuals.
Then I compared all the demarcation programs on larger data sets and found that in most cases ES2 maintained a high level of accuracy.
\emph{Bacillus} sequences analysis.
ES2 still lags behind the competition in terms of running time.
However, there was a dramatic improvement over ES1.

%Concluding analaysis
From these results, we can confidently say that ES2 continues to be the most accurate demarcation algorithm on larger generated datasets. However, we should use a combination of approaches, each lending strengths to the other's weaknesses, adding robustness to microbial ecosystem analysis~\cite{bohannan2003new}

\section{For the future}
%Might not need this section, then again some repetition might be good?
We have found that there are many areas we would like to improve on ES2.
There is an ongoing development process.
Parallelization using OpenMPI will lead to the greatest speed dividends in the near future.
However, an important emerging bottleneck is binning space usage.
We are considering, various approaches to the problem that include parallelization and algorithm switches.
Also we have identified several issues with the automatic demarcation program leading to inflated $npop$ prediction in the presence of paraphyletic groups.
And as usual speed has become an issue for demarcation.

Even with these ideas for improvement we aim to release a production ready version of ES2 soon.
Demarcation not limited by input size is an attractive goal, but we are not quite there yet.
We are continuing the development process in pursuit of bacterial species understanding that will eventually lead to practical applications.

