\chapter{Improvements of Ecotype Simulation}

\begin{shadequote}
Harder, Better, Faster, Stronger \par\emph{Daft Punk}
\end{shadequote}

\section{Improvements}
ES1 set hard input size limits of 2000 sample sequences of at most 3000 nucleotides~\cite{koeppel2008identifying}.
However, I doubt it was ever run with even close to that many sequences.
When run on input sizes of greater than only one hundred sequences time became a limiting factor.
In fact previous work showed that while ES1 had superior demarcation accuracy to its contemporaries, ES1 could not compete in run-time comparisons.
And as we increase the sample size running time increases dramatically.
See Table \ref{tab:ES1speed}.
This issue is clearly a priority for improvement.

To solve the problem we came up with two approaches.
First, modify the main ES algorithm driving the program.
Second, reorganize execution to take advantage of large computer clusters (parallelization).
I will focus mainly on modifications to the ES algorithm, which are prominently featured in ES2, and briefly address my colleague, Lingyuan Ke's, approach to (parallelization) which will be featured in future Ecotype Simulation iterations.
Finally, I will discuss future plans for improving the Ecotype Simulation software family.


\subsection*{Ecotype Simulation 2.0}
This is the version of ES I thoroughly test in chapter chapter four in comparison to other demarcating programs.
While the changes are few in number, we believe that they will increase the practicality of using ES approach to understanding microbial diversity.

\begin{table}
 %\begin{tabular}{| l | l | l | l |}
 \begin{tabular}{| c | c | c | c |}
  \hline
  Algorithm & 20 sequences & 30 sequences & 50 sequences \\ \hline
  ES & 69.8 & 384 & 2390 \\
  AdaptML & 1.54 & 1.57 & 1.64 \\
  GMYC & 0.201 & 0.292 & 0.549 \\
  BAPS & 4.80 & 5.15 & 6.12 \\
  \hline
 \end{tabular}
 \caption[ES1 run-time compared to other demarcation programs.]{The run speed is measured in seconds (reprinted from \protect\cite{carlo})}
 \label{tab:ES1speed}
\end{table}

\subsubsection*{Key algorithmic changes}
Backwards and forward simulation takes up most of ES1 run time.
The simulation process is run many times throughout the entire typical program execution.
If we can come up with a way to reduce the time complexity of simulations, great speed improvements are possible.

For ES2 we run the backwards simulation of node coalescence exactly the same.
However, once we have the evolutionary history scaffold we can use properties of ultra-metric phylogenetic trees.
Ultra-metric trees have the same distance between root to tip for all organisms, and usually are made under the assumptions of a molecular clock.
By the length between two nodes we can determine time between organisms.
Based on this fact performing binning is unnecessary.
ES2 can conduct a linear pass through the backwards scaffold directly comparing it to the observed sequence identity graph, checking for success and failure with all precision levels.
This insight removes an $O(n^3)$ factor, immediately quickening the algorithm.

\subsubsection*{Demarcation}
In addition to core algorithm changes we made changes to the auto-demarcating program.
Instead of demarcation confidence intervals, ES2 runs the downhill-simplex (Hillclimbing) method on each subtree as we recursively descend through the phylogeny.
Normally our Bruteforce method must be run before Hillclimbing, in this case we use results from Hillclimbing as a seed for the closest ancestor node.
Then at each child node we use Hillclimbing results from the parent node, bypassing running Bruteforce for each subtree.
This maintains a high level of accuracy without compromising speed.

%THINK ABOUT HOW TO ORDER THIS SECTION AND MAKE HEADERS
\section{Future Improvements}
%\subsection*{Ecotype Simulation 3.0}
\subsection*{Binning}
\subsubsection*{Complete linkage clustering}
\subsubsection*{Various implementations}
\subsubsection*{Minimizing space usage}
\subsection*{Parallelization}
\subsubsection*{OpenMP approach} %WIKI EXACT OPERATION OF OPENMP!!!
The OpenMP API is a commonly used shared-memory parallelism approach designed for C, C++, and Fortran programs.
In Fortran the programmer adds comments (known as directives) to specify OpenMP behavior.
These directives implicitly or explicitly define, perhaps guide, the execution of multiple threads as parallel programs

An OpenMP enriched program begins as a single thread of execution.
Whenever a thread encounters a parallel construct the thread creates a team of sub-threads, generates a set of tasks, and then declares itself master of the team.
Only the master thread resumes execution beyond the end of the parallel construct.
The program can specify any number of parallel constructs.

All threads have access to the same memory so they can retrieve variables, this is called a shared-memory model.
Also, each thread can specify private memory unreachable to other threads.
We use shared and private clause keywords to identify the respective paradigm.
%HERE LING REFERENCES AN ARTICLE

%Do I need this section, the one below that is.
\subsubsection*{Implementation in Fortran90}
Explain why we select this loop. %Ling includes a lot code here, but I don't think that'll be necessary.

I think it has something to do with a low-level vs high-level trade off. Don't want too much or either to maximize efficiency. Also we need a point that will make sense to split the task up.

%LING SETS UP A SECTION for describing specific sections of the OpenMP implementation. This may be more useful than going over code issues. Or we may just skip it and say to refer to Ling's thesis for an in depth explanation.

\subsubsection*{Random number generation}
%Problem subsections? Again we may not need to include this section (Random number generation)

%Solution

\subsubsection*{Parallelized tests}
%I don't think I should include much for this section. Again maybe outline the results but refer reader to Ling's Thesis for in depth details.

\subsubsection*{Setup}
\subsubsection*{Tests}
